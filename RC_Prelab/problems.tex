\documentclass{article}
\usepackage{circuitikz}

\begin{document}\noindent

\begin{enumerate}
\item
Consider the following simple circuit:

\begin{center}
  \begin{circuitikz} \draw
    ( 0, 0) node[anchor=south]{$V_{in}$}
            to[C, l=$C$, o-*]            ( 4, 0  ) 
            to[R, l=$R$]                 ( 4,-2  ) node[ground]{}
    ( 4, 0) to[short, -o]                ( 6, 0  ) node[anchor=south]{$V_{out}$};
  \end{circuitikz}
\end{center}

Let the input voltage $V_{in}$ be a sinusoidally varying function with amplitude
$V_0$ and angular frequency $\omega$.

\begin{enumerate}
  \item 
  Calculate the gain $g$ and phase shift $\phi$ for the output voltage
  relative to the input voltage.  
  \item Plot $g$ and $\phi$ as a function of
  $\omega / \omega_0$ where $\omega_0 = 1 / RC$. For each of these functions,
  use the combination of linear or logarithmic axes for g and for $\phi$ that
  you think are most appropriate.
\end{enumerate}

\item
Consider the following simple circuit:
\begin{center}
  \begin{circuitikz} \draw
    ( 0, 0) node[anchor=south]{$V_{in}$}
            to[R, l=$R$, o-*]            ( 4, 0  )
            to[L, l=$L$]                 ( 4,-2  ) node[ground]{}
    ( 4, 0) to[short, -o]                ( 6, 0  ) node[anchor=south]{$V_{out}$};
  \end{circuitikz}
\end{center}
Let the input voltage $V_{in}$ be a sinusoidally varying function with amplitude
$V_0$ and angular frequency $\omega$.

\begin{enumerate}
  \item Calculate the gain $g$ and phase shift $\phi$ for the output voltage
    relative to the input voltage.
  \item Plot $g$ and $\phi$ as a function of $\omega / \omega_0$ where $\omega_0
    = R/L$. For each of these functions, use the combination of linear or
    logarithmic axes for $g$ and for $\phi$ that you think are most appropriate.
\end{enumerate}

\pagebreak
\item Consider the following not-so-simple circuit:
\begin{center}
  \begin{circuitikz} \draw
    ( 0, 0) node[anchor=south]{$V_{in}$}
            to[R, l=$R$, o-*]           ( 4, 0  )
            to[short, *-*]              ( 4,-1  )
            to[short, *-]               ( 3,-1  )
            to[C, l_=$C$]               ( 3,-4  )  
            to[short, -*]                ( 4,-4  )
            to[short, *-]               ( 4,-4.5)  node[ground]{}
    ( 4,-1) to[short]                   ( 5,-1  )
            to[L, l=$L$]                ( 5,-4  )
            to[short, -*]               ( 4,-4  )
    ( 4, 0) to[short, -o]               ( 6, 0  )  node[anchor=south]{$V_{out}$};
  \end{circuitikz}
\end{center}

\begin{enumerate}
  \item What is the gain $g$ for very low frequencies $\omega$? What is the gain
    for very high frequencies? Remember that capacitors act like dead shorts and
    open circuits at high and low frequencies, respectively, and inductors
    behave in just the opposite way.
  \item At what frequency do you suppose the gain of this circuit is maximized?
  \item Using the rules for impedance and the generalized voltage divider,
    determine the gain $g(\omega)$ for this circuit and show that your answers
    to (a) and (b) are correct.
\end{enumerate}

\pagebreak
\begin{figure}[h!]
\begin{center}
\begin{circuitikz} \draw
  node[ground]{} ( 0, 0) to[sinusoidal voltage source] ( 0, 4  )
                         to[C, l=$1\mu F$]             ( 4, 4  )
                         to[R, l=$100\Omega$]          ( 4, 0.5)
                         to[short]                     ( 0, 0.5);
\end{circuitikz}
\caption{RC Circuit}
\label{c:RC}
\end{center}
\end{figure}

\item
For the capacitor in Circuit \ref{c:RC}, calculate the reactance for all five of
the following frequencies:10 Hz, 100 Hz, 1,000 Hz, 10,000 Hz, 100,000 Hz.

\item
Using elementary SI units, determine the units of capacitive reactance
and show that they are equivalent to ohms.

\item
Predict the minimum impedance for Circuit \ref{c:RC}. Explain your reasoning.

\begin{figure}[h!]
\begin{center}
\begin{circuitikz} \draw
  node[ground]{} ( 0, 0) to[sinusoidal voltage source] ( 0, 4  )
                         to[L, l=$44 mH$]             ( 4, 4  )
                         to[R, l=$100\Omega$]          ( 4, 0.5)
                         to[short]                     ( 0, 0.5);
\end{circuitikz}
\caption{LR Circuit}
\label{c:LR}
\end{center}
\end{figure}

\item For the inductor in Circuit \ref{c:LR}, calculate the reactance for all
  five of the following frequencies: 10 Hz, 100 Hz, 1,000 Hz, 10,000 Hz, and
  100,000 Hz.
\item Using elementary SI units determine the units of inductive reactance and
  show that they are equivalent to ohms.
\item Predict the minimum impedance for Circuit \ref{c:LR}. Explain your
  reasoning.

\begin{figure}[h!]
\begin{center}
\begin{circuitikz} \draw
  node[ground]{} ( 0, 0) to[sinusoidal voltage source] ( 0, 4  )
                         to[C, l=$1 \mu F$]            ( 2, 4  )
                         to[L, l=$44 mH$]              ( 4, 4  )
                         to[R, l=$100\Omega$]          ( 4, 0.5)
                         to[short]                     ( 0, 0.5);
\end{circuitikz}
\caption{LRC Circuit}
\label{c:LRC}
\end{center}
\end{figure}

\item Predict the minimum impedance for Circuit \ref{c:LRC}. Explain your
  reasoning.
\item Predict the resonant frequency for Circuit \ref{c:LRC}. Explain your
  reasoning
\item Calculate the impedance, $Z$, for circuit \ref{c:LRC} at the following
  five frequencies: 10 Hz, 100 Hz, 1,000 Hz, 10,000 Hz and 100,000 Hz. Next
  calculate the peak to peak current at these five frequencies, when the applied
  voltage is 1 volt peak to peak. Remember, for a series circuit $I_{pp} =
  \frac{V_{applied pp}}{Z}$.
\item Using the numbers from your calculations in Question 9 above, draw a
  practice plot of the current in circuit 3 versus the frequency. See the third
  paragraph in the Procedure section below for information on how to construct
  this plot.

\end{enumerate}
\end{document}


